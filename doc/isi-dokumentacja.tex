% najpierw jest wstep, załączenie potrzebnych pakietów itp.
\documentclass[a4paper, 10pt]{article}

%polskie znaki
\usepackage[polish]{babel}
\usepackage[utf8]{inputenc}
\usepackage[OT4]{fontenc}

%wieksze mozliwosci zmiany wygladu strony, pakiet do wstawiania linków
\usepackage{geometry}
\usepackage{ulem}
\RequirePackage{url}

% ladne wciecia akapitow i odstepy, mozna wykasowac wedle uznania;)
\setlength{\parindent}{0cm}
\setlength{\parskip}{3mm plus1mm minus1mm}

%mniejsze marginesy
\geometry{verbose,a4paper,tmargin=2.4cm,bmargin=2.4cm,lmargin=2.4cm,rmargin=2.4cm}
\usepackage{graphicx} % wstawianie obrazkow


%%%%%%%%%%%%%%%%%%%%%%%%%%%%%%%%%%%%%%%%%%%%%%%%

\title{{\bf {Inteligentne systemy informacyjne }} \\ {\large Dokumentacja projektu}}
\date{\today}
\author{Filip Nabrdalik}

%%%%%%%%%%%%%%%%%%%%%%%%%%%%%%%%%%%%%%%%%%%%%%%%
\begin{document}
\bibliographystyle{abbrv}
%%%%%%%
\null  % Empty line
\nointerlineskip  % No skip for prev line
\vfill
\let\snewpage \newpage
\let\newpage \relax
\maketitle %wstawienie tytulu, daty i autora
\let \newpage \snewpage
\vfill
\break % page break
%%%%%%%%%%%%%%%%%%%%%%%%%%%%%

\tableofcontents

\newpage






\section{Treść zadania}

{\bf{Zadanie 17}}

{\it }


\section{Algorytmy}



\section{Wyniki}
\section{Wnioski}







%BIBLIOGRAFIA
\nocite{*}
\bibliography{bibliografia}


\end{document}


